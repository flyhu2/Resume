%%%%%%%%%%%%%%%%%
% This is an sample CV template created using altacv.cls
% (v1.7.2, 28 August 2024) written by LianTze Lim (liantze@gmail.com). Compiles with pdfLaTeX, XeLaTeX and LuaLaTeX.
%
%% It may be distributed and/or modified under the
%% conditions of the LaTeX Project Public License, either version 1.3
%% of this license or (at your option) any later version.
%% The latest version of this license is in
%%    http://www.latex-project.org/lppl.txt
%% and version 1.3 or later is part of all distributions of LaTeX
%% version 2003/12/01 or later.
%%%%%%%%%%%%%%%%

%% Use the "normalphoto" option if you want a normal photo instead of cropped to a circle
% \documentclass[10pt,a4paper,normalphoto]{altacv}

\documentclass[10pt,a4paper,ragged2e,withhyper]{altacv}
%% AltaCV uses the fontawesome5 and simpleicons packages.
%% See http://texdoc.net/pkg/fontawesome5 and http://texdoc.net/pkg/simpleicons for full list of symbols.

% Change the page layout if you need to
\geometry{left=1.25cm,right=1.25cm,top=1.1cm,bottom=1.4cm,columnsep=0.5cm}

% The paracol package lets you typeset columns of text in parallel
\usepackage{paracol}
\usepackage{ctex}

\usepackage{xcolor}
\usepackage{longfbox}
\usepackage{tcolorbox}
\usepackage{ulem}

% Change the font if you want to, depending on whether
% you're using pdflatex or xelatex/lualatex
% WHEN COMPILING WITH XELATEX PLEASE USE
% xelatex -shell-escape -output-driver="xdvipdfmx -z 0" sample.tex
\ifxetexorluatex
  % If using xelatex or lualatex:
  \setmainfont{Roboto Slab}
  \setsansfont{Lato}
  \renewcommand{\familydefault}{\sfdefault}
\else
  % If using pdflatex:
  \usepackage[rm]{roboto}
  \usepackage[defaultsans]{lato}
  % \usepackage{sourcesanspro}
  \renewcommand{\familydefault}{\sfdefault}
\fi

% Change the colours if you want to
\definecolor{SlateGrey}{HTML}{2E2E2E}
%\definecolor{LightGrey}{HTML}{666666}
\definecolor{LightGrey}{HTML}{444444}
\definecolor{DarkPastelRed}{HTML}{450808}
\definecolor{PastelRed}{HTML}{8F0D0D}
\definecolor{GoldenEarth}{HTML}{E7D192}
\colorlet{name}{black}
\colorlet{tagline}{PastelRed}
\colorlet{heading}{DarkPastelRed}
\colorlet{headingrule}{GoldenEarth}
\colorlet{subheading}{PastelRed}
\colorlet{accent}{PastelRed}
\colorlet{emphasis}{SlateGrey}
\colorlet{body}{LightGrey}

% Change some fonts, if necessary
\renewcommand{\namefont}{\Huge\rmfamily\bfseries}
\renewcommand{\personalinfofont}{\footnotesize}
\renewcommand{\cvsectionfont}{\LARGE\rmfamily\bfseries}
\renewcommand{\cvsubsectionfont}{\large\bfseries}


% Change the bullets for itemize and rating marker
% for \cvskill if you want to
\renewcommand{\cvItemMarker}{{\small\textbullet}}
\renewcommand{\cvRatingMarker}{\faCircle}
% ...and the markers for the date/location for \cvevent
% \renewcommand{\cvDateMarker}{\faCalendar*[regular]}
% \renewcommand{\cvLocationMarker}{\faMapMarker*}


% If your CV/résumé is in a language other than English,
% then you probably want to change these so that when you
% copy-paste from the PDF or run pdftotext, the location
% and date marker icons for \cvevent will paste as correct
% translations. For example Spanish:
% \renewcommand{\locationname}{Ubicación}
% \renewcommand{\datename}{Fecha}


%% Use (and optionally edit if necessary) this .tex if you
%% want to use an author-year reference style like APA(6)
%% for your publication list
% \input{pubs-authoryear.tex}

%% Use (and optionally edit if necessary) this .tex if you
%% want an originally numerical reference style like IEEE
%% for your publication list
\input{pubs-num.tex}

%% sample.bib contains your publications
\addbibresource{sample.bib}

\begin{document}
%\name{周飞虎$~~$(2025届$~$计算机专业)}
\name{周飞虎$~~$(计算机科学与技术)}
%\tagline{意向岗位: $~$ISP/Camera/Display/影像/画质等$~$图像相关岗位}
%\tagline{意向岗位: \uline{~~~~~~~~~~~~~~~~~~~~~~~~~~~~~~~~~~}$~~~$意向城市: \uline{~~~~~~~~~~~~~~~~~~~~~~~~~~~~~~~}}
\tagline{意向岗位: \uline{~~ISP算法工程师~~}$~~~~$意向城市: \uline{~珠海、成都、武汉~}}
%\tagline{意向岗位: \uline{~~~~算法研究和应用~~~~}$~~~~$意向城市: \uline{~~~~成都、深圳~~~}}

%\tagline{意向岗位: \uline{~~~~~科研及专业技术岗~~~~~}$~~~~$意向城市: \uline{~~四川·成都~~}}

%% You can add multiple photos on the left or right
\photoR{3.0cm}{me02}
%\photoL{2.5cm}{me02}
%\photo[rectangle,edge,right]{me02}
\personalinfo{%
  % Not all of these are required!
  % You can add your own with \printinfo{symbol}{detail}
  % \email{flyhu2@126.com}
  \faIcon{envelope}{ $~$flyhu2@126.com$~~~$}
  \faIcon{phone-square-alt}{ $~$17683724669$~~~$}
  % \phone{17683724669}
  \faIcon{map-marker-alt}{ $~$湖北·荆州$~~~$}
  % \location{湖北省·荆州市$~~~~~~~~~~~~~$ 1998年01月}
  \faBirthdayCake{ $~$1998.01$~~~$}
  \faAddressCard{$~$中共党员}
%   \github{} % I'm just making this up though.
%   \orcid{orcid.org/0000-0000-0000-0000} % Obviously making this up too. If you want to use this field (and also other academicons symbols), add "academicons" option to \documentclass{altacv}
}


\makecvheader
%% Depending on your tastes, you may want to make fonts of itemize environments slightly smaller
% \AtBeginEnvironment{itemize}{\small}

%% Set the left/right column width ratio to 6:4.
\columnratio{0.63}

% Start a 2-column paracol. Both the left and right columns will automatically
% break across pages if things get too long.142, 14, 13
\begin{paracol}{2}
\vspace{-1.2mm}
\cvsection{教育经历\hfill{\fontsize{12}{11}\selectfont 2016.09 -- 2025.06}}
\cvevent{广西大学 \tcbox[on line, rounded corners, colback=teal!30!white, colframe={rgb:red,25;green,255;blue,105}, boxrule=0pt, arc=3pt, left=1pt, right=1pt, top=0pt, bottom=0pt]{\fontsize{10}{11}\selectfont 211} | 硕士:计算机科学与技术 | {\hfill 2022 -- 2025}}{}{}{}

\begin{itemize}
	\item {绩点与排名}: {\color{red}{4.25}}/5.0 $~$ | $~$ {\color{red}{11}}/64 $~$ | $~$ {\color{red}{17}\%}
	\item 研究方向: 暗光图像增强({\textbf{LLIE}}),超分辨率({\textbf{SR}}),{\textbf{AI-ISP}},计算摄影等
	\item 主修课程: 计算机数学 | 数字图像处理与计算机视觉 | 高级人工智能等
\end{itemize}

\divider

\cvevent{海南大学 \tcbox[on line, rounded corners, colback=teal!30!white, colframe={rgb:red,25;green,255;blue,105}, boxrule=0pt, arc=3pt, left=1pt, right=1pt, top=0pt, bottom=0pt]{\fontsize{10}{11}\selectfont 211} | 学士:通信工程 | {\hfill 2016 -- 2020}}{}{}{}
\begin{itemize}
	\item 绩点与排名: {\color{red}{3.79}}/4.0 $~$ | $~$  {\color{red}{7}}/111 $~$ | $~$ {\color{red}{6.3}\%}
	\item 英语及其它: 大学英语六级 ({\color{red}CET-6}),普通话二级乙等,C1驾照等
	\item 主修课程: 通信原理 | 信号与系统 | 数字信号处理 | MATLAB及应用等  \\ 
\end{itemize}

\cvsection{项目经历\hfill {\fontsize{12}{11}\selectfont 2022.09 -- 2025.06}}

%\cvevent{京东方科技集团股份有限公司\hfill (BOE,面板行业龙头企业)}{PE \& 不良机理研究部 / 研究员\hfill 北京·大兴}{}{}
%\begin{itemize}
%	\item 负责对显示模组的不良现象进行详细分析,并反馈给设计和工艺部门以改进,
%	如:亮暗点、线不良,绑定不良,Flicker不良等。
%	\item 负责LCD/OLED模组的光学/电学性指标能测试,如亮度均匀性、
%	对比度、色域容积、TFT响应曲线,灰阶响应时间,高低温信耐性等。
%\end{itemize}
\cvevent{低光照条件下的联合图像增强与目标检测方法研究\footnote{《项目地址:https://github.com/flyhu2/DarkSR》}}{国家自然科学基金 / 科研骨干}{}{}
\begin{itemize}
	\item \textbf{项目描述}:增强低光照条件下的图像质量以及提高目标检的测精度。
	\item \textbf{工作职责}:(1)负责Low-Level的任务:低照度下的图像复原/增强;
	$~~~~~$(2)理解RAW图像处理(即ISP Pipline),并用深度学习实现AI-ISP。
	\item \textbf{项目成果}:(1)一篇软著登记(排名第一,授权时间:2024.02)
	%$~~~~~$(2)一篇软著:《基于光照融合的暗光图像增强算法软件》(排名第一)
	$~~~~~$(2)\textbf{一篇{\color{red}{~CCF-C~}}国际会议论文\footnote{《Joint Image Super-resolution and Low-light Enhancement in the Dark》第一作者}({\color{red}{ACCV 2024}} 亚洲计算机视觉顶会)}
	\item \textbf{论文贡献}:(1)提出了用于夜间图像超分辨率的数据集:DarkSR;
	$~~~~~$(2)提出了一个基于RAW和sRGB图像双输入的网络结构:JSLNet;
	$~~~~~$(3)在JSLNet中使用小波变换在频域完成特征融合并实现增强任务;
	$~~~~~$(4)大量实验证明了提出的JSLNet在主客观评价上优于SOTA方法,\\
  并且在真实场景中具有更好的泛化性能(使用数据集之外数据推理)。
\end{itemize}


\cvsection{工作经历\hfill {\fontsize{12}{11}\selectfont 2020.07 -- 2021.08}}
%\cvevent{京东方科技集团股份有限公司\hfill{\fontsize{10}{11}\selectfont {\textbf{\color{PastelRed}PE \& 不良机理研究部 / 研究员}}  }}{}{}{}
\cvevent{京东方科技集团股份有限公司(北京·大兴)}{中央研究院 / PE \& 不良机理研究部 / 研究员}{}{}

\begin{itemize}
	\item 负责对显示模组的不良现象进行详细分析,并反馈给设计和\\
  工艺部门以改进,如:亮暗点、线不良,绑定不良,Flicker不良等。
	\item 负责LCD/OLED模组的光学/电学性指标能测试,如亮度均匀性、\\
	对比度、色域容积、TFT响应曲线,灰阶响应时间,高低温信耐性等。
\end{itemize}

\medskip



% use ONLY \newpage if you want to force a page break for
% ONLY the current column

\switchcolumn


\vspace{-0.0mm}
\cvsection{奖励荣誉}

\begin{itemize}
	\item {\fontsize{9}{11} \selectfont 广西大学“2024年度优秀学生干部”}
	\item {\fontsize{9}{11} \selectfont 广西大学“2023年度优秀共青团员”}
	\item {\fontsize{9}{11} \selectfont 广西大学“2024年研究生二等学业奖学金”}
	\item {\fontsize{9}{11} \selectfont 广西大学“2023年研究生一等学业奖学金”}
	\item {\fontsize{9}{11} \selectfont 广西大学“2022年研究生二等学业奖学金”}
\end{itemize}
\divider
\begin{itemize}
	\item {\fontsize{9}{11} \selectfont 海南大学“2020届优秀毕业生”}
	\item {\fontsize{9}{11} \selectfont 海南大学“三好学生” × 2次}
	\item {\fontsize{9}{11} \selectfont 国家励志奖学金 × 2次}
	\item {\fontsize{9}{11} \selectfont 海南大学“优秀共青团团干部”}
	\item {\fontsize{9}{11} \selectfont 海南大学“最具创新精神和实践能力大学生”}
	\item {\fontsize{9}{11} \selectfont 全国大学生电子设计大赛海南赛区:一等奖}
	\item {\fontsize{9}{11} \selectfont 海南省大学生电子设计大赛本科组:三等奖}
\end{itemize}

\cvsection{校园经历}

\cvtag{班长 \& 团支书 }
\cvtag{校级学生组织干事}\\
\cvtag{学习委员}
\cvtag{计电学院党建宣传部部长}



\cvsection{技能评价}
%\divider\smallskip


%\cvtag{责任心}\cvtag{团队合作精神}\\
\cvtag{扎实的英语读写能力,熟练阅读英文文献}\\
%\cvtag{掌握Python语言,Pytorch深度学习框架}\\
%\cvtag{具有半导体行业工作经历,熟悉CMOS电路}\\%
\cvtag{具有半导体行业工作经历,熟悉显示原理}\\
\cvtag{掌握图像增强以及复原的基本原理和算法}\\
%\cvtag{个人学历记录博客,粉丝数>2300}\\
\cvtag{擅于借助AI工具解决工作和生活中的问题}\\
%\cvtag{在处理复杂或重复的任务时仍然具有耐心}\\
\cvtag{Python}
\cvtag{Pytorch}
\cvtag{MATLAB}
\cvtag{C/C++}
\\
\cvtag{office 办公套件}
%\cvtag{Excel}
%\cvtag{PowerPoint}
\cvtag{ISP Pipline}
\cvtag{个人博客}


%\cvskill{German}{3.5} %% Supports X.5 values.

%% Yeah I didn't spend too much time making all the
%% spacing consistent... sorry. Use \smallskip, \medskip,
%% \bigskip, \vspace etc to make adjustments.
\medskip
\cvsection{兴趣爱好}

% Adapted from @Jake's answer from http://tex.stackexchange.com/a/82729/226
% \wheelchart{outer radius}{inner radius}{
% comma-separated list of value/text width/color/detail}

\centerline
{\wheelchart{1.2cm}{0.0cm}{%
  4/10em/accent!70/{摄影},
  3/10em/accent!60/{阅读},
  2.5/10em/accent!50/{无人机},
  1.5/10em/accent!40/{运动}
}}
\smallskip
%{\fontsize{10}{11} \selectfont \textbf{热爱摄像和摄影,对画质有较高的追求,\\有志于长期从事和影像/画质有关的方向}}
%{\fontsize{10}{11} \selectfont \textbf{乐于接受新事物,保持好奇和乐观的心态\\}}
{\fontsize{10}{11} \selectfont \textbf{$~$有志于长期从事和图像/视觉有关的方向\\}}
%{\fontsize{10}{11} \selectfont \textbf{$~~$乐于接受新事物,兴趣广泛,乐观积极\\}}

\end{paracol}
\end{document}

