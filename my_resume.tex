%%%%%%%%%%%%%%%%%
% This is an sample CV template created using altacv.cls
% (v1.7.2, 28 August 2024) written by LianTze Lim (liantze@gmail.com). Compiles with pdfLaTeX, XeLaTeX and LuaLaTeX.
%
%% It may be distributed and/or modified under the
%% conditions of the LaTeX Project Public License, either version 1.3
%% of this license or (at your option) any later version.
%% The latest version of this license is in
%%    http://www.latex-project.org/lppl.txt
%% and version 1.3 or later is part of all distributions of LaTeX
%% version 2003/12/01 or later.
%%%%%%%%%%%%%%%%

%% Use the "normalphoto" option if you want a normal photo instead of cropped to a circle
% \documentclass[10pt,a4paper,normalphoto]{altacv}

\documentclass[10pt,a4paper,ragged2e,withhyper]{altacv}
%% AltaCV uses the fontawesome5 and simpleicons packages.
%% See http://texdoc.net/pkg/fontawesome5 and http://texdoc.net/pkg/simpleicons for full list of symbols.

% Change the page layout if you need to
\geometry{left=1.25cm,right=1.25cm,top=1.1cm,bottom=1.4cm,columnsep=0.5cm}

% The paracol package lets you typeset columns of text in parallel
\usepackage{paracol}
\usepackage{ctex}

\usepackage{xcolor}
\usepackage{longfbox}
\usepackage{tcolorbox}

% Change the font if you want to, depending on whether
% you're using pdflatex or xelatex/lualatex
% WHEN COMPILING WITH XELATEX PLEASE USE
% xelatex -shell-escape -output-driver="xdvipdfmx -z 0" sample.tex
\ifxetexorluatex
  % If using xelatex or lualatex:
  \setmainfont{Roboto Slab}
  \setsansfont{Lato}
  \renewcommand{\familydefault}{\sfdefault}
\else
  % If using pdflatex:
  \usepackage[rm]{roboto}
  \usepackage[defaultsans]{lato}
  % \usepackage{sourcesanspro}
  \renewcommand{\familydefault}{\sfdefault}
\fi

% Change the colours if you want to
\definecolor{SlateGrey}{HTML}{2E2E2E}
%\definecolor{LightGrey}{HTML}{666666}
\definecolor{LightGrey}{HTML}{444444}
\definecolor{DarkPastelRed}{HTML}{450808}
\definecolor{PastelRed}{HTML}{8F0D0D}
\definecolor{GoldenEarth}{HTML}{E7D192}
\colorlet{name}{black}
\colorlet{tagline}{PastelRed}
\colorlet{heading}{DarkPastelRed}
\colorlet{headingrule}{GoldenEarth}
\colorlet{subheading}{PastelRed}
\colorlet{accent}{PastelRed}
\colorlet{emphasis}{SlateGrey}
\colorlet{body}{LightGrey}

% Change some fonts, if necessary
\renewcommand{\namefont}{\Huge\rmfamily\bfseries}
\renewcommand{\personalinfofont}{\footnotesize}
\renewcommand{\cvsectionfont}{\LARGE\rmfamily\bfseries}
\renewcommand{\cvsubsectionfont}{\large\bfseries}


% Change the bullets for itemize and rating marker
% for \cvskill if you want to
\renewcommand{\cvItemMarker}{{\small\textbullet}}
\renewcommand{\cvRatingMarker}{\faCircle}
% ...and the markers for the date/location for \cvevent
% \renewcommand{\cvDateMarker}{\faCalendar*[regular]}
% \renewcommand{\cvLocationMarker}{\faMapMarker*}


% If your CV/résumé is in a language other than English,
% then you probably want to change these so that when you
% copy-paste from the PDF or run pdftotext, the location
% and date marker icons for \cvevent will paste as correct
% translations. For example Spanish:
% \renewcommand{\locationname}{Ubicación}
% \renewcommand{\datename}{Fecha}


%% Use (and optionally edit if necessary) this .tex if you
%% want to use an author-year reference style like APA(6)
%% for your publication list
% \input{pubs-authoryear.tex}

%% Use (and optionally edit if necessary) this .tex if you
%% want an originally numerical reference style like IEEE
%% for your publication list
\input{pubs-num.tex}

%% sample.bib contains your publications
\addbibresource{sample.bib}

\begin{document}
%\name{XXX$~~$(2025届$~$计算机专业)}
\name{XXX$~$(专业/其他不写会很空)}
%\tagline{意向岗位: $~$ISP/Camera/Display/影像/画质等$~$图像相关岗位}
\tagline{意向岗位: $\underline{~~~~~~~~~~~~~~~~~~~~~~~~~~~~~~~~~~~~~~~~~~~}$意向城市: $\underline{~~~~~~~~~~~~~~~~~~~~~~~~~}$}

%% You can add multiple photos on the left or right
\photoR{3.0cm}{Globe_High}
%\photoL{2.5cm}{me02}
%\photo[rectangle,edge,right]{me02}
\personalinfo{%
  % Not all of these are required!
  % You can add your own with \printinfo{symbol}{detail}
  \faIcon{envelope}{ $~$xxxxx@xxx.com$~~~$}
  \faIcon{phone-square-alt}{ $~$1xxxxxxxxxx$~~~$}
  \faIcon{map-marker-alt}{ $~$湖北·xxx$~~~$}
  \faBirthdayCake{ $~$200x.xx$~~~$}
  \faAddressCard{$~$政治面貌}
%   \github{} % I'm just making this up though.
%   \orcid{orcid.org/0000-0000-0000-0000} % Obviously making this up too. If you want to use this field (and also other academicons symbols), add "academicons" option to \documentclass{altacv}
}


\makecvheader
%% Depending on your tastes, you may want to make fonts of itemize environments slightly smaller
% \AtBeginEnvironment{itemize}{\small}

%% Set the left/right column width ratio to 6:4.
\columnratio{0.63} %调节双栏占比

% Start a 2-column paracol. Both the left and right columns will automatically
% break across pages if things get too long.142, 14, 13
\begin{paracol}{2}
\vspace{-1.2mm}
\cvsection{教育经历\hfill{\fontsize{12}{11}\selectfont 20xx.09 -- 20xx.06}}
\cvevent{XX大学 \tcbox[on line, rounded corners, colback=teal!30!white, colframe={rgb:red,25;green,255;blue,105}, boxrule=0pt, arc=3pt, left=1pt, right=1pt, top=0pt, bottom=0pt]{\fontsize{10}{11}\selectfont 211} | 硕士:某某专业 | {\hfill 20xx -- 20xx}}{}{}{}

\begin{itemize}
	\item {绩点与排名}: {\color{red}{4.xx}}/5.0 $~$ | $~$ {\color{red}{x}}/xx $~$ | $~$ {\color{red}{xx}\%}
	\item 研究方向: 具体方向({\textbf{英文缩写}}),如超分辨率({\textbf{SR}}),{\textbf{AI-ISP}},计算摄影等
	\item 主修课程: 主要课程 | 体现匹配度 | 被问到要会 | 不需要太多巴拉巴拉等  \\ 
\end{itemize}

\divider

\cvevent{XX大学 \tcbox[on line, rounded corners, colback=teal!30!white, colframe={rgb:red,25;green,255;blue,105}, boxrule=0pt, arc=3pt, left=1pt, right=1pt, top=0pt, bottom=0pt]{\fontsize{10}{11}\selectfont 211} | 学士:某某专业 | {\hfill 20xx -- 20xx}}{}{}{}
\begin{itemize}
	\item 绩点与排名: {\color{red}{3.xx}}/4.0 $~$ | $~$  {\color{red}{x}}/xxx $~$ | $~$ {\color{red}{xx}\%}
	\item 英语及其它: 大学英语六级 ({\color{red}CET-6}),xxxxxxx,xxxxx等
	\item 主修课程: 主要课程 | 体现匹配度 | 被问到要会 | 不需要太多巴拉巴拉等  \\ 
\end{itemize}

\cvsection{项目经历\hfill {\fontsize{12}{11}\selectfont 20xx.09 -- 20xx.06}}


\cvevent{具体的项目名称XXXXXXXXXXXXXXXXXX\footnote{《项目地址:https://github.com/项目的主页》}}{国家自然科学基金 / 科研骨干}{}{}
\begin{itemize}
	\item \textbf{项目描述}:根据实际内容补充完整,并调整整体地美观,填充填充。
	\item \textbf{工作职责}:(1)根据实际内容补充完整,并调整整体地美观,填充;
	$~~~~~$(2)根据实际内容补充完整,并调整整体地美观,填充填充填充。
	\item \textbf{项目成果}:(1)根据实际内容补充完整,并调整整体地美观,填充;
	%$~~~~~$(2)一篇软著:《基于光照融合的暗光图像增强算法软件》(排名第一)
	$~~~~~$(2)\textbf{一篇{\color{red}{~CCF-X~}}国际会议论文\footnote{《论文名称xxxxxxxxxxxxxxxxxxxxxxxxxxxxxxxxxxx》第一作者}({\color{red}{XXXX 2024}} 中文名称XXXXXXXX)}
	\item \textbf{论文贡献}:(1)根据实际内容补充完整,并调整整体地美观,填充;
	$~~~~~$(2)根据实际内容补充完整,并调整整体地美观,填充填充填充;
	$~~~~~$(3)根据实际内容补充完整,并调整整体地美观,填充填充填充;
	$~~~~~$(4)根据实际内容补充完整,并调整整体地美观,填充填充填充\\
	根据实际内容补充完整,并调整整体地美观,填充填充填充填充填充。
\end{itemize}


\cvsection{工作经历\hfill {\fontsize{12}{11}\selectfont 20xx.xx -- 20xx.xx}}
\cvevent{XXXXXXXXX股份有限公司(湖北·XX)}{XX组织 / 具体部门 / 岗位名称}{}{}

\begin{itemize}
	\item 根据实际内容补充完整,并调整整体地美观,填充填充填充填充填充\\
	根据实际内容补充完整,并调整整体地美观,填充填充填充填充填充。
	\item 根据实际内容补充完整,并调整整体地美观,填充填充填充填充填充\\
	根据实际内容补充完整,并调整整体地美观,填充填充填充填充填充。
\end{itemize}

\medskip



% use ONLY \newpage if you want to force a page break for
% ONLY the current column

\switchcolumn


\vspace{-0.0mm}
\cvsection{奖励荣誉}

\begin{itemize}
	\item {\fontsize{9}{11} \selectfont XX大学“某某荣誉1”,根据情况调整长短}
	\item {\fontsize{9}{11} \selectfont XX大学“某某荣誉2”,根据情况调整长短}
	\item {\fontsize{9}{11} \selectfont XX大学“某某荣誉3”,根据情况调整长短}
	\item {\fontsize{9}{11} \selectfont 某某比赛/竞赛等:X等奖,根据情况调整}
\end{itemize}
\divider
\begin{itemize}
	\item {\fontsize{9}{11} \selectfont YY大学“某某荣誉1”,根据情况调整长短}
	\item {\fontsize{9}{11} \selectfont YY大学“某某荣誉2”,根据情况调整长短}
	\item {\fontsize{9}{11} \selectfont YY大学“某某荣誉3”,根据情况调整长短}
	\item {\fontsize{9}{11} \selectfont YY大学“某某荣誉4”,根据情况调整长短}
	\item {\fontsize{9}{11} \selectfont YY大学“某某荣誉5”,根据情况调整长短}
	\item {\fontsize{9}{11} \selectfont 某某比赛/竞赛等:X等奖,根据情况调整}
	\item {\fontsize{9}{11} \selectfont 某某比赛/竞赛等:X等奖},根据情况调整
\end{itemize}

\cvsection{校园经历}

\cvtag{班长 \& 团支书 }
\cvtag{校级学生组织干事}\\
\cvtag{XXX委员}
\cvtag{XXXXXXXXXXXXXXXX部长}



\cvsection{技能评价}
%\divider\smallskip

\cvtag{扎实的英语读写能力,熟练阅读英文文献}\\
\cvtag{具有XXXXXXXXXXXXXXX,熟悉XXXXXX}\\
\cvtag{掌握XXXXXXXXXXXXXXXXXXXXXXXXXX}\\
\cvtag{擅于XXXXXXXXXXXXXXXXXXXXXXXXXX}\\
\cvtag{Python}
\cvtag{Pytorch}
\cvtag{MATLAB}
\cvtag{C/C++}
\\
\cvtag{XXXX其他1}
%\cvtag{Excel}
%\cvtag{PowerPoint}
\cvtag{XXXX其他2}
\cvtag{XXX其他3}


%\cvskill{German}{3.5} %% Supports X.5 values.

%% Yeah I didn't spend too much time making all the
%% spacing consistent... sorry. Use \smallskip, \medskip,
%% \bigskip, \vspace etc to make adjustments.
\medskip
\cvsection{兴趣爱好}

% Adapted from @Jake's answer from http://tex.stackexchange.com/a/82729/226
% \wheelchart{outer radius}{inner radius}{
% comma-separated list of value/text width/color/detail}

\centerline
{\wheelchart{1.2cm}{0.0cm}{%
  4/10em/accent!70/{爱好1},
  3/10em/accent!60/{爱好2},
  2.5/10em/accent!50/{爱好3},
  1.5/10em/accent!40/{爱好4}
}}
\smallskip
{\fontsize{10}{11} \selectfont \textbf{$~$有志于长期从事和XXX/XXX有关的方向\\}}
%{\fontsize{10}{11} \selectfont \textbf{$~~$乐于接受新事物,兴趣广泛,乐观积极\\}}

\end{paracol}
\end{document}
